\documentclass{beamer}

\mode<presentation>
{
  \usetheme{CambridgeUS}
  \setbeamercovered{transparent}
}

\usepackage[english]{babel}
\usepackage[latin1]{inputenc}
\usepackage{times}
\usepackage[T1]{fontenc} 
% Or whatever. Note that the encoding and the font should match. If T1
% does not look nice, try deleting the line with the fontenc.
\usepackage{amsmath}

\newcommand{\linespace}{\vskip 0.25cm}

\definecolor{MyForestGreen}{rgb}{0,0.7,0} 
\newcommand{\tableemph}[1]{{#1}}
\newcommand{\tablewin}[1]{\tableemph{#1}}
\newcommand{\tablemid}[1]{\tableemph{#1}}
\newcommand{\tablelose}[1]{\tableemph{#1}}

\definecolor{MyLightGray}{rgb}{0.6,0.6,0.6}
\newcommand{\tabletie}[1]{\color{MyLightGray} {#1}}

% The text in square brackets is the short version of your title and will be used in the
% header/footer depending on your theme.
\title[Mobile Security]{Gait Recognition in Mobile Security}

% Sub-titles are optional - uncomment and edit the next line if you want one.
% \subtitle{Why does sub-tree crossover work?} 

% The text in square brackets is the short version of your name(s) and will be used in the
% header/footer depending on your theme.
\author[Ottomoeller]{Chase R. Ottomoeller}

% The text in square brackets is the short version of your institution and will be used in the
% header/footer depending on your theme.
\institute[U of Minn, Morris]
{
  Division of Science and Mathematics \\
  University of Minnesota, Morris \\
  Morris, Minnesota, USA
}

% The text in square brackets is the short version of the date if you need that.
\date[December '14, SS, Morris] % (optional)
{December 6, 2014 \\ Senior Seminar, Morris}

% Delete this, if you do not want the table of contents to pop up at
% the beginning of each subsection:
\AtBeginSection[]
{
  \begin{frame}<beamer>
    \frametitle{Outline}
    \tableofcontents[currentsection, hideothersubsections]
  \end{frame}
}

\begin{document}

\begin{frame}
  \titlepage
\end{frame}

% For a 20-25 minute senior seminar talk you probably want something like:
% - Two or three major sections (other than the summary).
% - At *most* three subsections per section.
% - Talk about 30s to 2min per frame. So there should probably be between
%   15 and 30 frames, all told.

\section*{Overview}

\subsection*{The Big Picture}

\begin{frame}
  \frametitle{The Big Picture}
  
  \begin{columns}
  \begin{column}{0.6\textwidth}
  What is Mobile Security?
     \begin{itemize}
     \item Information Storage
  	  \item Device Access 
  	  \linebreak
     \end{itemize}
     
  How is mobile security evolving?
     \begin{itemize}
     \item No More Passwords
	  \item Something You Are
	  \item Unobtrusive Access
%	\item What is Gait Recognition?
%	\item How is Gait Recognition useful?
     \end{itemize}
  \end{column}
  \begin{column}{0.4\textwidth}
   \includegraphics[width=0.95\textwidth]{Illustrations/mobileSecurity.jpg}
       \\
%    \only{\tiny{Bluedrakon \\ \url{http://mobilebuzz.guru/wp-content/uploads/2014/06/Mobile-Security.png} }}
  \end{column}
  \end{columns}
\end{frame}

\subsection*{Outline}

\begin{frame}
  \frametitle{Outline}
  \tableofcontents[hideallsubsections]
\end{frame}





\section[Mobile Security]{Background}

\subsection{Biometrics}
\begin{frame}
  \frametitle{Biometrics}

  \begin{columns}
  \begin{column}{0.4\textwidth}
  \begin{itemize}
    \item Biometrics 
  	\linebreak
  	\item Gait Recognition
  	\linebreak
	\item Why Gait is Better
  \end{itemize}
  \end{column}
  \begin{column}{0.8\textwidth}
   \includegraphics[width=0.8\textwidth]{Illustrations/allbiometrics.jpg}
       %http://www.smc2012.org/images/jain3-1.jpg
  \end{column}
  \end{columns}
\end{frame}

\subsection{Two Methods}
\subsection{}
\begin{frame}
  \frametitle{Fixed Method Approach}
  \begin{columns}
  \begin{column}{0.4\textwidth}
  \begin{itemize}
    \item Phone Clipped to Waist 
  	\linebreak
  	\item Walked Down 18.5 Meter Hallway
  	\linebreak
  	\item Separated into "Walks"
  \end{itemize}
  \end{column}
  \begin{column}{0.8\textwidth}
   \includegraphics[width=0.8\textwidth]{Illustrations/gaitpatterns.png}
       \\
  \end{column}
  \end{columns}
\end{frame}

\subsection{}
\begin{frame}
  \frametitle{Unfixed Method Approach}

  \begin{columns}
  \begin{column}{0.6\textwidth}
  \begin{itemize}
    \item Phone in more natural location (pocket, handbag, backpack)
  	\linebreak
  	\item Performed in Real-world Environments
  	\linebreak
  	\item Separated Into Frames
  \end{itemize}
  \end{column}
  \begin{column}{0.7\textwidth}
   \includegraphics[width=0.6\textwidth]{Illustrations/frame.jpg}
       %http://www.smc2012.org/images/jain3-1.jpg
  \end{column}
  \end{columns}
\end{frame}





\section[Preprocessing the data]{Preprocessing The Data}

\subsection{What is Preprocessing?}
\begin{frame}
  \frametitle{Preprocessing}

  \begin{columns}
  \begin{column}{0.5\textwidth}
  \begin{itemize}
    \item Separates walking data from other noise
    \linebreak
    \item Walks VS Frames
  \end{itemize}
  \end{column}

\begin{column}{0.02\textwidth}
\end{column}

\begin{column}{0.5\textwidth}

\includegraphics[width=0.91\textwidth]{Illustrations/preprocessingexample.png}

\includegraphics[width=0.91\textwidth]{Illustrations/frame.jpg}

\end{column}  
  
  
  \end{columns}
\end{frame}

\subsection{Fixed Method Preprocessing}
\begin{frame}
  \frametitle{Linear Interpolation and Zero Normalization}
  \begin{columns}
  \begin{column}{0.4\textwidth}
  \begin{itemize}
    \item Walk Extraction 
  	\linebreak
  	\item Linear Interpolation (curve fitting)
  	\linebreak
	\item Zero Normalization 
  \end{itemize}
  \end{column}  
  
\begin{column}{0.5\textwidth}

\includegraphics[width=0.91\textwidth]{Illustrations/onegaitcycle.png}

\includegraphics[width=0.91\textwidth]{Illustrations/linear.png}

\end{column}   
  
  \end{columns}
\end{frame}

%\subsection{Unfixed Preprocessing}
%\begin{frame}
%  \frametitle{Unfixed Method Overview}
%  \begin{columns}
%  \begin{column}{1\textwidth}
%  \begin{itemize}
%  
%  	\item Three Experiments:
% 	\begin{itemize}
% 		\item Training Walking Detector
% 		\linebreak
% 		\item Evaluating Supervised Training Data
% 		\linebreak
% 		\item Unsupervised Training
% 	\end{itemize}
% 	
%  \end{itemize}
%  \end{column}
%  \begin{column}{0.7\textwidth}
%   %\includegraphics[width=0.7\textwidth]{Illustrations/ab.png}  
%  \end{column}
%  \end{columns}
%\end{frame}
\subsection{Unfixed Method Preprocessing}
\begin{frame}
  \frametitle{Framing}
  \begin{columns}
  \begin{column}{0.65\textwidth}
  	\begin{itemize}
  		\item Separating Data into Equal Sections    
  		\linebreak
  		\item Frame Length: 5.12 seconds       
  		\linebreak
  		\item Each Frame contains 512 Samples
  		\linebreak
  		\item Stationary frames are dropped                
  	\end{itemize}
  \end{column}
  \begin{column}{0.6\textwidth}
   \includegraphics[width=0.6\textwidth]{Illustrations/frame.jpg}
       \\
  \end{column}
  
  \end{columns}
\end{frame}

\begin{frame}
  \frametitle{Projection}
  \begin{columns}
  \begin{column}{.7\textwidth}
  
  	\begin{itemize}
  		\item Each sample is projected onto a global coordinate system (sample = {x, y, and z})
  		\linebreak
  		\item Estimating direction of gravity with changes in x, y, and z. 
  		\linebreak
  		\item Each sample is split into two vectors: 
  			\begin{itemize}
  				\item Vertical (x)
  				\item Horizontal (y, z)
  				\linebreak
			\end{itemize}
  		\item Frame dropped if orientation is changed  
  	\end{itemize}
  
  \end{column}
  
  \begin{column}{0.7\textwidth}
   \includegraphics[width=0.5\textwidth]{Illustrations/global.jpg}
       \\
  \end{column}
  
  \end{columns}
\end{frame}




\section[Feature Extraction]{Feature Extraction}

\subsection{What is Feature Extraction?}
\begin{frame}
  \frametitle{What is Feature Extraction?}
  \begin{columns}
  \begin{column}{.9\textwidth}
  \begin{itemize}
  	\item Feature extraction separates "walking" cycles from "non-walking" cycles
  \end{itemize}
  \end{column}
  \begin{column}{0.6\textwidth}
   %\includegraphics[width=0.6\textwidth]{Illustrations/rawgaitdata.png}
       \\
  \end{column}
  \end{columns}
\end{frame}

\subsection{Fixed Method Feature Extraction}
\begin{frame}
  \frametitle{Fixed Method Feature Extraction}
  \begin{columns}
  \begin{column}{.5\textwidth}
  \begin{itemize}
  	\item Four Steps:
  	\begin{itemize}
  		\item Cycle Length Estimation
  		\linebreak
  		\item Cycle Detection
  		\linebreak
  		\item Cycle length normalization
  		\linebreak
  		\item Omitting Unusual Cycles
  	\end{itemize}
  \end{itemize}
  \end{column}
  \begin{column}{0.6\textwidth}
   %\includegraphics[width=0.6\textwidth]{Illustrations/rawgaitdata.png}
       \\
  \end{column}
  \end{columns}
\end{frame}

\begin{frame}
  \frametitle{Cycle Length Estimation}
  \begin{columns}
  \begin{column}{.5\textwidth}
  \begin{itemize}
  	\item Estimate cycle lengths by computing Minimum Salience Vectors
  	\linebreak
  	\item Minimum Salience Vector
  		\begin{itemize}
  			\item Contains one entry for each data point
  			\item Each entry is the count of data values between the current value and following smaller value
  		\end{itemize}
  \end{itemize}
  \end{column}
    \begin{column}{0.7\textwidth}
   \includegraphics[width=0.8\textwidth]{Illustrations/svector.png}
       \\
  \end{column}
  \end{columns}
\end{frame}

\begin{frame}
  \frametitle{Cycle Detection}
  \begin{columns}
  \begin{column}{.5\textwidth}
  \begin{itemize}
  	\item Detecting Individual Cycles
  	\item Start of each cycle is located using the entry with the greatest value
  	\item Spike around points 750, 1150, 1450, 1650 
  	\item Long cycles are split again using the same method
  \end{itemize}
  \end{column}
    \begin{column}{0.7\textwidth}
   \includegraphics[width=0.8\textwidth]{Illustrations/svector.png}
       \\
  \end{column}
  \end{columns}
\end{frame}

\begin{frame}
  \frametitle{Cycle Length Normalization}
  \begin{columns}
  \begin{column}{.55\textwidth}
  \begin{itemize}
  	\item The distance of each cycle is measured from the start of one cycle to the start of the following. 
  	\linebreak
  	\item Cycles need to be of a set length for later Gait Analysis
  	\linebreak
  	\item Linear Interpolation 
  \end{itemize}
  \end{column}
    \begin{column}{0.7\textwidth}
   \includegraphics[width=0.7\textwidth]{Illustrations/nonnormalized.png}
       \\
  \end{column}
  \end{columns}
\end{frame}

\begin{frame}
  \frametitle{Omitting Unusual Cycles}
  \begin{columns}
  \begin{column}{.5\textwidth}
  \begin{itemize}
  	\item Deleting Unusual Gait Cycles
  	\linebreak
  	\item Dynamic Time Warping (DTW): An algorithm used to measure similarity between two sequences
  	\linebreak
  	\item Cycles with an acceleration half that of the average are dropped
  \end{itemize}
  \end{column}
    \begin{column}{0.6\textwidth}
   \includegraphics[width=0.9\textwidth]{Illustrations/DTW.jpg}
       \\
  \end{column}
  \end{columns}
\end{frame}

%////////////////////////////////////////////////////////////////////////////////////////////////////////////
\subsection{Unfixed Feature Extraction}

\begin{frame}
  \frametitle{Unfixed Method Feature Extraction}
  \begin{columns}
  \begin{column}{.5\textwidth}
  \begin{itemize}
  	\item Three Steps:
  	\begin{itemize}
  		\item Feature Extraction I
  		\linebreak
  		\item Walking Detection
  		\linebreak
  		\item Feature Extraction II
  	\end{itemize}
  \end{itemize}
  \end{column}
  \begin{column}{0.6\textwidth}
   %\includegraphics[width=0.6\textwidth]{Illustrations/rawgaitdata.png}
       \\
  \end{column}
  \end{columns}
\end{frame}



\begin{frame}
\frametitle{Feature Extraction I}
	\begin{itemize}
		\item Determine differences between "walking" and "non-walking"
		\linebreak
		\item Walking 1-2Hz vs Running >3Hz 
		\linebreak
		\item These features are used in Walking Detection
		
	\end{itemize}
	
\end{frame}

\begin{frame}
\frametitle{Walking Detection}
	\begin{itemize}
		\item Three classifications using a decision tree:
			\begin{itemize}
				\item Walking: 1-2Hz
				\linebreak
				\item Non-Walking: >3Hz (running, biking, in moving vehicle)
				\linebreak
				\item Random Movements: >0Hz (transitional movements, short spikes)
				\linebreak
			\end{itemize}
			
		\item Cycles labeled as walking move onto the next step
	\end{itemize}
\end{frame}

\begin{frame}
\frametitle{Feature Extraction II}
	\begin{itemize}
		\item Once Walking Detection confirms that the frame contains walking data, more relevant features are extracted
		\linebreak
		
		\item Some features extracted using Autocorrelation
		
		
%		\item Two sets of features extracted:
%		
%		\begin{itemize}
%			\item Fundamental Frequency of Movement
%			\item Autocorrelation Features
%		\end{itemize}



	\end{itemize}
\end{frame}

%\begin{frame}
%\frametitle{Fundamental Frequencies}
%	\begin{itemize}
%		\item This first set of features computed in this stage is the Compressed Sub-band Cepstral Coefficients (CSCC) 
%		\item CSCC based of off feature set for audio analysis.
%		\item CSCC evolves 3 steps:
%		\begin{itemize}
%			\item[1)] The energy spectrum is computed using the Fast Fourier Transform (FFT)
%			\item[2)] Then the energy spectrum is mapped into 26 bands
%			\item[3)] The discrete cosine transform of the sub-band energy is taken to form a 12-dimension vector representation
%		\end{itemize}
%	\end{itemize}
%\end{frame}

\begin{frame}
\frametitle{Autocorrelation}
 \begin{columns}
  \begin{column}{.6\textwidth}
  \begin{itemize}
		\item Useful to find periodicity and cadence of the gait
		\item Collecting data from a phone inside a pocket
		\item Jostling of the phone can create spikes
		\item Segmentation, like minimum salience vectors, cannot be used
		\item Autocorrelation can reveal features even with noise
  \end{itemize}
  \end{column}
  \begin{column}{0.65\textwidth}
   \includegraphics[width=0.6\textwidth]{Illustrations/autocorrelationcd.png}
       \\
  \end{column}
  \end{columns}  
  
\end{frame}
%/////////////////////////////////////////////////////////////////////////////////////////////////////////////////////////////////////////
\section[Gait Analysis]{Gait Classification}

\subsection{Overview}
\begin{frame}
\frametitle{What is Gait Classification?}
 \begin{columns}
  \begin{column}{.9\textwidth}
  \begin{itemize}
  		\item Gait Classification determines if the user is "genuine" or an "impostor"
  		\linebreak
		\item Fixed Method Gait Classification
			\begin{itemize}
				\item Template-based 
				\item Machine Learning 
				\linebreak
			\end{itemize}
			
		\item Unfixed Method Gait Classification
			\begin{itemize}
				\item Gausian Mixture Model-Universal Background Model
			\end{itemize}
  \end{itemize}
  \end{column}
  \begin{column}{0.7\textwidth}
   %includegraphics[width=0.7\textwidth]{Illustrations/autocorrelation.png}
       \\
  \end{column}
  \end{columns}  
\end{frame}

\subsection{Fixed Method Gait Classification}
\begin{frame}
\frametitle{Template-based}
 \begin{columns}
  \begin{column}{.8\textwidth}
  \begin{itemize}
		\item Feature Cycle (the cycle with the lowest DTW distance)
		\linebreak
		\item Probe Cycles (the remaining cycles)
		\linebreak
		\item After computing probe and reference cycles for all walks two classes are made:
			\begin{itemize}
			\item Genuine
			\item Impostor
			\linebreak
			\end{itemize}
		
		\item Genuine and Impostor are made by comparing the DTW distance of all the reference and probe cycles					
		\linebreak
		\item 50\% of the Probe cycles must vote genuine		
  \end{itemize}
  \end{column}
  \begin{column}{0.7\textwidth}
   %\includegraphics[width=0.7\textwidth]{Illustrations/autocorrelation.png}
       \\
  \end{column}
  \end{columns}  
\end{frame}

\begin{frame}
\frametitle{Machine Learning}
 \begin{columns}
  \begin{column}{.8\textwidth}
  \begin{itemize}
  		\item Data is split into two groups:
  			\begin{itemize}
  				\item Training (80\%)
  				\item Testing (20\%)
  			\end{itemize}
		\item Support Vector Machines (SVMs) are used for biometric classification
		\item A SVM finds a hyperplane that linearly separates data into two classes: genuine and impostor
  \end{itemize}
  \end{column}
  \begin{column}{0.7\textwidth}
   %\includegraphics[width=0.7\textwidth]{Illustrations/autocorrelation.png}
       \\
  \end{column}
  \end{columns}  
\end{frame}

\begin{frame}
\frametitle{Machine Learning}
 \begin{columns}
  \begin{column}{.6\textwidth}
  \begin{itemize}
  		\item The data is not usually linearly separable. Therefore, a kernel function is used.
		\item A kernel function maps non linearly separable data to a high dimension space. 
		\item These data points are now compared to the Testing data set. Again, the class with the maximum votes wins.
  \end{itemize}
  \end{column}
  \begin{column}{0.6\textwidth}
   \includegraphics[width=0.6\textwidth]{Illustrations/svm.jpg}
       \\
  \end{column}
  \end{columns}  
\end{frame}

\subsection{Unfixed Method Gait Classification}
 \begin{frame}
\frametitle{Unfixed Method }
 \begin{columns}
  \begin{column}{.9\textwidth}
  \begin{itemize}
  		\item The use of more than one training model is used to help classify gait cycles
  		
		\item Universal Background Model (UBM) is used to train a large source of data
	
		\item User's gait model is generated relating the odds of one event to the odds of another
		
		\item Maximum-a-Posteriori (MAP) is used to adjust Gaussian components and mixture weight to personalize the UBM model
		
		\item The current user's gait cycle is compared to the personalized UBM model and either accepts or rejects.

		\item MAP is also used by recording false negatives
  \end{itemize}
  \end{column}
  \begin{column}{0.2\textwidth}
   %\includegraphics[width=0.7\textwidth]{Illustrations/autocorrelation.png}
       \\
  \end{column}
  \end{columns}  
\end{frame}

\section[Conclusion]{Conclusion}
\begin{frame}
\frametitle{Conclusions}
	\begin{itemize}
		\item The unfixed method is developed more and is better for real-life situations
		\linebreak
		\item Given time, the fixed method can perform just as well as the unfixed method
	\end{itemize}
	
\end{frame}






\section*{References}

\begin{frame} 
	\frametitle{References} 
%	
%	\begin{thebibliography}{lskdjf}
%	
%	\bibitem{McPhee:2009:gecco}
%N.~F. McPhee, E.~Crane, S.~Lahr, and R.~Poli.
%\newblock Developmental Plasticity in Linear Genetic Programming.
%\newblock In G\"unther Raidl, \emph{et al}, editors, {\em GECCO '09}, pages 1019--1026, Montr\'eal, Qu\'ebec, Canada, 2009.
%	
%	\bibitem{citeulike:3452411}
%	R.~Poli and N.~McPhee.
%\newblock A linear estimation-of-distribution {GP} system.
%\newblock In M.~O'Neill, \emph{et al}, editors, {\em EuroGP 2008}, volume
%  4971 of {\em LNCS}, pages 206--217, Naples,
%  26-28 Mar. 2008. Springer.
%  
%  	\end{thebibliography}
%	
%	\linespace
%	\begin{center}
%	See the GECCO '09 paper for additional references.
%	\end{center}
\end{frame} 

\end{document}


